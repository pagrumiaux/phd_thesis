\begin{abstract}
\addtocontents{toc}{\vspace{-0.5cm}}
\addchaptertocentry{\abstractname}

Sound source localization (SSL) is a subtask of audio scene analysis that has challenged researchers for more than four decades. Traditional methods (e.g., MUSIC or GCC-PHAT) impose strong assumptions on the sound propagation, number of active sources and/or signal content, which makes them vulnerable to adverse acoustic phenomena, such as reverberation and noise. Recently, data-driven models -- and particularly deep neural networks – have shown increased robustness in noisy and reverberant environments. However, their performance is still seriously degraded in the presence of multiple sound sources, especially when their number is unknown. Moreover, source detection and localization in real-life use-cases, where the latency is an important criterion, is still an open research problem. 
 
In this thesis, we focus on speaker detection and localisation in office/domestic indoor environments, using multichannel Ambisonics recordings, with the emphasis on low-latency performance. First, we propose to use deep neural networks (DNNs) to estimate the number of speakers (NoS) in a multichannel mixture. We propose a model that is capable to count up to five speakers, with a relatively high accuracy, at the short-term-frame resolution. We also provide a performance analysis of this model depending on several hyperparameters, which gives interesting insights on its behavior. Second, we explore the capabilities of a multichannel audio signal representation called time-domain velocity vector (TDVV), akin to relative impulse response in the present spherical harmonics domain, as a novel type of input features of DNNs for detection/localization tasks. Next, we address multi-speaker localization, by first improving upon a state-of-the-art convolutional recurrent neural network (CRNN) with a substantial gain in accuracy. We also examine the potential of self-attention-based neural networks for multi-speaker localization, as these models are known to be suitable for other audio processing tasks due to their capability to capture both short- and long-term dependencies in the input signal. Furthermore, we investigate the use of the estimated NoS, provided by our speaker counting neural network, to improve our speaker localization CRNN. We show experimentally that using the estimated NoS leads to more robust multi-speaker localization than the classical threshold-based direction of arrival (DoA) estimation. Moreover, we show the interest of injecting the NoS information as an additional input feature for the localization neural network. Finally, we explore multi-task neural architectures to estimate both the NoS and speaker DoAs at the same time.

\end{abstract}