\begin{resume}
\addchaptertocentry{Résumé}

La localisation de sources sonores est une sous-tâche de l'analyse de scènes sonores qui a défié les chercheurs pendant plus de quatre décennies. Les méthodes traditionnelles (e.g., MUSIC ou GCC-PHAT) imposent des hypothèses fortes sur la propagation du son, le nombre de sources actives et/ou le contenu du signal, ce qui les rend vulnérables à des phénomènes acoustiques adverses tels que la réverbération ou le bruit. Récemment, les méthodes basées sur les données -- et particulièrement les réseaux de neurones profonds – ont montré une plus grande robustesse dans les environnements réverbérants et bruités. Cependant, leur performance est toujours sensiblement dégradée en présence de plusieurs sources sonores, notamment quand leur nombre est inconnu. De plus, la détection et la localisation de sources pour des usages pratiques, où la latence joue un rôle important, est toujours un sujet de recherche ouvert.

Dans cette thèse, nous nous intéressons à la détection et à la localisation de locuteurs dans des environnements domestiques, en utilisant des enregistrements ambisoniques multicanaux, avec un accent sur une performance à basse latence. Tout d'abord, nous proposons d'utiliser des réseaux de neurones profonds (DNN, pour deep neural network) pour estimer le nombre de locuteurs (NoS, number of sources) dans un mélange multicanal. Notre modèle est capable de compter jusqu'à cinq locuteurs, avec une précision relativement grande, pour une résolution à la trame. Nous proposons également une analyse de la performance du modèle en fonction de certains hyperparamètres, ce qui fournit des informations intéressantes sur son comportement. Ensuite, nous explorons les capacités d'une représentation d'un signal audio multicanal appelée vecteur vélocité dans le domaine temporel (TDVV, time-domain velocity vector), qui est analogue à la réponse impulsionnelle relative dans le domaine des harmoniques sphériques, en tant que nouvelle représentation d'entrée de DNNs pour la localisation/détection. Par la suite, nous nous penchons sur la localisation de plusieurs locuteurs en commençant par améliorer un réseau de neurones convolutif et récurrent (CRNN, convolutional recurrent neural network) de l'état de l'art avec un gain important en précision. Puis nous examinons le potentiel des mécanismes de self-attention pour la localisation de plusieurs locuteurs, alors que ces modèles sont connus pour être adaptés à d'autres tâches de traitement audio étant donnée leur capacité à capter les dépendances à court et long terme dans le signal d'entrée. En outre, nous investiguons l'utilisation du NoS estimé, fourni par notre réseau de neurones de comptage, pour améliorer le CRNN de localisation. Nous montrons expérimentalement qu'utiliser le NoS estimé donne plus de robustesse à la localisation multi-locuteur que la méthode de seuillage classiquement utilisée dans l'estimation de direction d'arrivée (DoA, direction of arrival). De plus, nous montrons l'intérêt d'injecter l'information du NoS en tant qu'entrée additionnelle pour le réseau de neurones de localisation. Finalement, nous explorons les architectures neuronales multi-tâches pour estimer le NoS et la DoA des locuteurs dans le même temps.
\end{resume}